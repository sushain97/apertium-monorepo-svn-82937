\documentclass{beamer}
\setbeamertemplate{navigation symbols}{}

\usepackage[english]{babel}
\usepackage{ucs} %sami letters
% \usepackage{amssymb} %mathematical
\usepackage[utf8x]{inputenc}
\usepackage[T1]{fontenc}
\usepackage{harvard}
\usepackage{multirow}
%\usepackage{rotating} 

%\usepackage{tikz}
%\usepackage{array}
%\usepackage{arydshln} %has to be after array
%\usepackage{multirow}
\usepackage{graphics}
\usepackage{tabularx} %specified width
\usepackage{tipa}
\usepackage{booktabs}
\usepackage{ctable} %loads booktable by default
\usepackage{colortbl}
%\usepackage{covington}
\usepackage{url}
%\usepackage[right=2.5cm,left=2.5cm,top=2cm,bottom=2cm]{geometry}
% \usepackage{bibtexlogo}
\usepackage{setspace}

\usepackage{fancyhdr}
\usepackage{linguex}


\usetheme{Montpellier} %Montpellier

\beamersetuncovermixins{\opaqueness<1>{25}}{\opaqueness<2->{15}}
\begin{document}
\title{Workshop i Nordsamisk-lulesamisk maskinoversetting}  
\author{Linda Wiechetek, Francis Tyers, Trond Trosterud}
\date{\today} 
\begin{frame}
\titlepage
\end{frame}

\begin{frame}\frametitle{Innhold}
\tableofcontents
\end{frame} 


\section{Buoris boahtem!} 
\begin{frame}\frametitle{Buoris boahtem!} 
\begin{itemize}
\item maskinoversetting i bilingual publisering (daglige aviser)
\item daglige aviser
\item skolebøker
\item flere tekster betyr også større etterspørsel
\end{itemize}
\end{frame}


%\subsection{Subsection no.1.1  }
\begin{frame}\frametitle{Forventninger}  
\begin{itemize}
\item Hva forventer du av en oversetting? \pause
\item Hva forventer du av maskinoversetting?
\end{itemize}

\end{frame}

\section{Maskinoversetting}
\begin{frame}\frametitle{Maskinoversetting}  
\begin{columns}
\begin{column}{5cm}
\begin{itemize}
\item<1-> Google
\item<3-> Babelfish
\item<5-> PROMT
\item<7-> Gramtrans
\end{itemize}
\vspace{3cm} 
\end{column}
\begin{column}{5cm}
\begin{overprint}
\scalebox{.2}[.2]{\includegraphics<2>{google.png}}
%\includegraphics<4>{PIC2}
%\includegraphics<6>{PIC3}
\scalebox{.2}[.2]{\includegraphics<6>{gramtrans.png}}
\end{overprint}
\end{column}
\end{columns}
\end{frame}


\begin{frame} 

\begin{exampleblock}{The Guardian 27.9.2009}
Last week, Iran said it was building a second uranium enrichment plant despite UN demands that it stops its development plans.
\end{exampleblock} \pause

\begin{exampleblock}{Google en-nb}
Sist uke sa \alert{??} at Iran var det å bygge et \alert{sekund} \alert{anriking av uran anlegg} til tross for FNs krav om at den stopper sine utviklingsplaner.
\end{exampleblock} \pause
% doesn't recognize the subject
% doesn't recognize the compound - anriking av uran anlegg

\begin{exampleblock}{Gramtrans en-nb}
Sist uke, sa Iran det holdt på å bygge et annen \alert{uranberigelsesanlæg} tross \alert{UN} krav \alert{som} det stopper sine utviklingsplaner.
\end{exampleblock}
% doesn't translate UN into FN

\begin{exampleblock}{Manuell oversetting}
Forrige uke sa Iran at de var i ferd med å bygge enda en urananrikingsfabrikk til tross for FNs krav om å stoppe utviklingsplanene.
\end{exampleblock}

\end{frame}

\begin{frame}
\begin{exampleblock}{Google nb-en}
Last week, Iran said that they were about to build another \alert{urananrikingsfabrikk} despite UN demands to halt development plans.
\end{exampleblock}

\begin{exampleblock}{Gramtrans nb-en}
\alert{Forrige} week Iran said that they were building still an \alert{urananrichingsfabrikk} in spite of the UN's demand for stopping the development plans.
\end{exampleblock}

\end{frame}

\section{Hva er god oversetting?} 
\begin{frame}\frametitle{Hva er god oversetting?}
\begin{itemize}
\item innhold vs. (kunstnerisk) budskap
\item ordrett vs. fri oversetting 
\item fremmedgjøring eller identifikasjon 
\end{itemize} 
\end{frame}

\begin{frame}
\frametitle{Hva er god oversetting?}
\begin{tabular}{|c|c|}
\hline
\textbf{Språk} & \textbf{Navn} \\
\hline
Italiensk &  Paolino Paperino  \\
\hline
Kroatisk &  Paško Patak  \\
\hline
Kinesisk & Tánglăo Yā \\
\hline
Litauisk & Ančiukas Donaldas \\
\hline
Islandsk & Andrés Önd \\
\hline
Nordsamisk & Vulle Vuojaš \\
\hline
Engelsk & Donald Duck \\
\hline
\end{tabular}
\end{frame}

\section{Hva er god maskinoversetting?}
%
\begin{frame}\frametitle{Hva er god maskinoversetting?}
%
\begin{exampleblock}{Genesis 11,9 (New International Version (NIV))}
That is why it was called Babel because there the LORD confused the language of the whole world. From 
there the LORD scattered them over the face of the whole earth.
\end{exampleblock}

\begin{exampleblock}{manuell oversetting}
Derfor kalte de den Babel. For der forvirret Herren all verdens tungemål, og derfra spredte Herren dem ut over hele jorden.
-- \url{http://www.bibel.no}
\end{exampleblock}

\begin{exampleblock}{Google}
Derfor ble det kalt Babel fordi det Herren forvirret språket i hele verden. Derfra spredte Herren dem over ansiktet til hele jorden.
\end{exampleblock}
\end{frame}
 
\begin{frame}
\frametitle{}
\begin{itemize}
\item textype: manual, newspapertexts
\item assimilation - dissemination
\end{itemize} 
\end{frame}

\begin{frame}
\frametitle{Uses of machine translation}

\begin{onlyenv}<1>
There are two main uses for machine translation, \emph{assimilation} and \emph{dissemination}:\\
~\\
\begin{table}
\begin{tabular}{l|l|l}
~  & Requirement & Not requirement\\
\hline
\multirow{3}{*}{Assimilation} & Understandability         & Syntactic \emph{correctness}\\
                              & \emph{Online} translation & Lexical \emph{correctness}\\
                              &                  & \\
\hline
\multirow{3}{*}{Dissemination} & Adequate syntactic transfer & Understandability \\
                               &  Predictable errors   & \emph{Online} translation \\
                               &  High accuracy (WER $\le$ 15\%)    & \\
\hline
\end{tabular}
\end{table}

\end{onlyenv}

\begin{onlyenv}<2->

- Useful for assimilation, not dissemination:\\
~\\
\emph{Erbyn heddiw mae Gaeleg yn dal yn iaith fyw yn y gogledd orllewin. }\\
\end{onlyenv}

\begin{onlyenv}<3->
By today a Scots Gaelic is tall in life language in the north west.\\
\end{onlyenv}

\begin{onlyenv}<4->
~\\
`Nowadays Scots Gaelic is still a living language in the north west.'\\ (WER = 43\%)\\
~\\
\end{onlyenv}

\begin{onlyenv}<5->

- Useful for dissemination, not assimilation:\\
~\\
\emph{Mae'r \alert<7>{mudiad} wedi derbyn canmoliaeth a beirniadaeth.}\\
\end{onlyenv}

\begin{onlyenv}<6->
The \alert<7>{migration} has received praise and criticism.\\
\end{onlyenv}

~\\
\begin{onlyenv}<7>
`The \alert<7>{organisation} has received praise and criticism.'\\ (WER = 14\%)\\
\end{onlyenv}

~\\

\end{frame}

\begin{frame}
\frametitle{}
\begin{itemize}
  \item lik antall setninger
  \item nærmeste oversetting av ord
  \item syntaktisk struktur skal være så nært som mulig
  \item PoS skal være like
  \item innhold
\end{itemize}
\end{frame}


\begin{frame}
\frametitle{What is machine translation}

  \begin{centering}

    {\Large rule-based vs. statistical}

    (what we know vs. what we can infer)

  \end{centering}

\end{frame}


\begin{frame}
\frametitle{The {\tt apertium-sme-smj} system}

\begin{itemize}
  \item \underline{Prototype} system -- basically proof-of-concept
  \item Focussed heavily on reuse of existing resources -- testbed for 
    integrating Giellatekno work with Apertium
  \begin{itemize}
    \item Morphological transducers for North and Lule Sámi
    \item Constraint Grammar disambiguator
    \item Transfer lexicon (bilingual dictionary) built largely automatically
  \end{itemize}
  \item Still under development
\end{itemize}

\end{frame}



\section{Kontrastiv lingvistikk} 
\begin{frame}\frametitle{Kontrastiv lingvistikk}
\begin{itemize}
\item Something about contrastive linguistics \pause
\item Things we found out about Lule Sámi vs. North Sámi \pause
\item Orthography - how was the bilingual dictionary made \pause
\item dat/duot/dat vs....\pause
\item Negation - tense marker in different elements \pause
\item SVO vs. SOV \pause
\item Case Loc vs. Ine/El \pause
\item Different PoS in some cases gullevaš vs. gullujiddje \pause
\item Why it can be interesting to work as a linguist -- being a discoverer \pause
\item Computational systems force you to be accurate -- when you need to be accurate you go into depth of linguistic questions
\end{itemize} 
\end{frame}


\end{document}
