\documentclass{beamer}
\usepackage[utf8x]{inputenc}
\usepackage[T1]{fontenc}
\usetheme{Copenhagen}

\title[Apertium Advanced Web Interface]{A first step towards interactivity and language tools convergence}

\author[Vié, Villarejo Muñoz, Farrús Cabeceran, O'Regan] % (optional, for multiple authors)
{Arnaud~Vié\inst{1} \and Luis~Villarejo~Muñoz\inst{2} \and Mireia~Farrús~Cabeceran\inst{2} \and Jimmy~O'Regan\inst{3}}
\institute
{
  \inst{1}%
  Informations Systems Engineering \\
  Grenoble INP - Ensimag \\
  {\tt arnaud.vie@ensimag.fr}
  \and
  \inst{2}%
  Learning Technologies Office \\
  Universitat Oberta de Catalunya \\
  {\tt lvillarejo@uoc.edu}, \\* 
  {\tt mfarrusc@uoc.edu}
  \and
  \inst{3}%
  Eolaistriu Technologies \\
  {\tt joregan@gmail.com}
}

\date{}

\begin{document}

\begin{frame}
\titlepage
\end{frame}

\begin{section}{Introduction}
\begin{frame}{Two Main Uses of MT}
\begin{description}
  \item[Assimilation] \hfill \\
  understanding the text
  \item[Dissemination] \hfill \\
  producing text for consumption
\end{description}
\end{frame}

\begin{frame}{Dissemination}
No MT system is perfect.

Post-editing is required to produce text fit for human consumption.
\end{frame}

\begin{frame}{Post-Edition}
MT Post-Edition is an active research area, with two main aims:

\begin{itemize}
\item Improve the translation.
\item Improve the MT system itself.
\end{itemize}
\end{frame}

\begin{frame}{Apertium and Post-Editing}
Apertium currently has {\it no} post-editing facilities.

This is an obvious disadvantage to the user, who must use other means
to improve the translation.

A less obvious disadvantage is to the developers of Apertium language
pairs, who miss out on user feedback that could be used to improve the
system.
\end{frame}

\begin{frame}{Two Types of Edits}
\begin{description}
  \item[Brief Edits] \hfill \\
  small, local changes (mostly syntactic)
  \item[Full Edits] \hfill \\
  larger changes (mostly stylistic)
\end{description}
\end{frame}

\begin{frame}{Full Edits}
The most famous example of a Full Edit system is Google Translate's
``Suggest a Better Translation'' feature.

Users click on a sentence, and type a replacement. Google (after checking
for a number of undesirables) later add this to their corpora for later
integration.
\end{frame}

\begin{frame}{Full Edits}
The Open Source Computer Assisted Translation tools Virtaal and OmegaT
both include the facility to have the current sentence translated
using (mostly) online MT services.

To a certain degree, they can be considered Open Source MT post-edition
tools, which allow Full Edits.

However, using a desktop system with online translations brings the 
disadvantages of desktop software to an online service (and vice versa).
\end{frame}

\begin{frame}{Brief Edits}
Ironically, Google Translate is also the most famous example of a
post-edition interface including Brief Edits.

The ``Suggest'' feature has recently been replaced with a simpler
feature that allows the user to click on an individual word, and
choose an alternative translation.
\end{frame}
\end{section}

\begin{section}{Project Goals}
\begin{frame}{Primary Goals}
\begin{itemize}
\item Fully online
\item Full Edits
\item Easy to integrate with existing installations
\item Extensible
\item Feedback
\item Integration with existing tools
\end{itemize}
\end{frame}

\begin{frame}{Fully online}
Most Apertium users use it via a web interface.
\end{frame}

\begin{frame}{Full edits}
The translator should be in control, not the interface
\end{frame}

\begin{frame}{Integration with Existing Installations}
Most current web interfaces to Apertium are written in PHP; this
interface was also written in PHP to make integration easier.
\end{frame}

\begin{frame}{Extensible}
The interface was designed to be a {\it framework} for post-edition.

Lacking current work in Apertium on automated post-edition, we cannot
know what future needs will be.

With the rapid pace of language pair development, we cannot know what
future language pairs will need from post-edition.

We don't assume.
\end{frame}

\begin{frame}{Feedback}
The interface should log user changes, for protential use in 
improving language pairs, and in developing automated post-edition tools.
\end{frame}

\begin{frame}{Integration with existing tools}
We should leverage other Open Source projects where appropriate.

Spell checking is provided by Aspell.

Grammar checking is provided by LanguageTool.
\end{frame}
\end{section}

\begin{section}{Reusing User Edits}
\begin{frame}{Translation Memory}
The most obvious way to reuse edits is by using Translation Memory.

Apertium already includes support for TMX.

Does not work particularly well below the sentence level.

The mALIGNa sentence alignment tool is integrated, to allow
creation of TMX files from edited sentences.
\end{frame}

\begin{frame}{Edit Logging}
Realtime keystroke logging.

Uses browser events.

Associated events are aggregated and ``promoted'': a deleted 
word is logged as a deleted word, not as a series of character 
deletions.
\end{frame}

\begin{frame}{Word Objects}

Each word is represented by an object containing:

\begin{itemize}
\item Current text (mutable)
\item Original text (immutable)
\item Reference to containing sentence object
\item Reference to containing document node
\item Position within the document node
\item Reference to previous and next word objects
\end{itemize}
\end{frame}

\begin{frame}{Sentence Objects}

Each sentence is represented by an object containing:

\begin{itemize}
\item References to its first and last words
\item References to previous and next sentence objects
\end{itemize}
\end{frame}

\begin{frame}{Logging}
Logging is invoked whenever editing occurs.

The logging module aggregates character level events and
promotes them to word level events (deletion, insertion)
or sentence level events (joining/splitting sentences)
as appropriate.
\end{frame}

\begin{frame}{Logging}
This aggregation is flexible: it may be modified to
aggregate events in different ways, to provide more
user feedback, etc., without changing the character-level
logging system, which performs the bulk of the work.
\end{frame}
\end{section}

\begin{section}{User Experience}
\begin{frame}{Formatted Documents}
Apertium operates as a series of programs in a Unix-like pipeline:
each program reads from standard input, and writes to standard output.

The first and last programs of a typical pipeline take care of removing
formatting information, and restoring it, respectively.

This formatting information is retained inline, escaped as
{\it superblanks}: extended text contained within braces,
which are treated as spaces by the translation components.
\end{frame}

\begin{frame}{Superblanks}

Similarly to the convention used in Translation Memory software,
superblanks are presented to the user as a set of colour-coded
(greyed), immutable strings in the interface, so the user can
know at all times where to move words in relation to formatting.
\end{frame}
\begin{frame}{Spell checking and Grammar checking}

All checking is performed server-side, for consistency, and to reduce 
administrative overhead.

When the user presses the ``Check for Mistakes'' button, spelling and 
grammar errors in the text are underlined (in red and blue, respectively).
The user may then click on a word to see any suggestions the server
may have to offer.

\end{frame}

\begin{frame}{External dictionaries}
Mistake suggestions may be accompanied by translations of the suggestions
provided by external web-based dictionaries, to help the user to find
the intended meaning.

Dictionaries are provided using OpenSearch XML files (used by the search
feature in the Firefox and Chrome browsers), for which a large number
of dictionaries have ready made configurations.
\end{frame}

\begin{frame}{Search and Replace}
The interface includes a search and replace feature.

In addition to the usual 'case sensitive'/'case insensitive' features,
the interface includes an 'Apply source case' option, which takes
case information from the original string and applies it to the
replacement string.
\end{frame}
\end{section}

\begin{section}{Future Possibilies}
\begin{frame}{Future/Ongoing Work}

The current interface has mostly been tested in Firefox: more
testing is required for other browsers.

More work is required to improve the response in the interface.

TMX use with Apertium' built-in support is not currently optimal; OmegaT
recently added an option to act as an automatic sentence translation system. 
We are investigating integrating it.

We are also looking into allowing After the Deadline to be used as an
alternative to LanguageTool.
\end{frame}
\end{section}

\begin{section}{End}
\begin{frame}{Thanks}
Questions?
\end{frame}

\begin{frame}{But first...}
Arnaud could not join us today, because of his studies. 

I hope you will join me in wishing him luck!
\end{frame}

\begin{frame}{Copying}
\begin{figure*}
  \caption{This presentation is Open Content}
     \includegraphics[width=\textwidth]{by-sa.png}
\end{figure*}

This presentation is licensed under the Creative Commons Attribution 
Share-Alike 3.0 Licence.

\end{frame}

\end{section}

\end{document}​
