\section{Development}

\subsection{Resources}
This language pair was developed with the
aid of on-line resources containing word definitions and flective
paradigms, such as \emph{Hrvatski jezični
  portal}\footnote{\url{http://hjp.srce.hr}} for the Serbo-Croatian side. For
the Slovenian side we used a similar online resource \emph{Slovar
  slovenskega knjižnega
  jezika},\footnote{\url{http://bos.zrc-sazu.si/sskj.html}} and the
\emph{Amebis Besana} flective
lexicon.\footnote{\url{http://besana.amebis.si/pregibanje/}}

The bilingual dictionary for the language pair was developed from scratch,
using the \emph{EUDict}\footnote{\url{http://eudict.com/}} online
dictionary and other online resources.
%\emph{Google Translate}\footnote{\url{http://translate.google.com/}}.

\subsection{Morphological analysis and generation}
The basis for this language pair are the morphological
lexicons for Serbo-Croatian (from
the language pair Serbo-Croatian--Macedonian, {\small{\tt apertium-hbs-mak}}) and Slovenian (from the
language pair Slovenian--Spanish, {\small{\tt apertium-slv-spa}}). Both
lexicons are written in the XML format of
\emph{lttoolbox}\footnote{\url{http://wiki.apertium.org/wiki/Lttoolbox}}
\cite{rojas2005construccion}, and were developed as parts of
their respective language pairs, during the Google Summer of Code
2011.\footnote{\url{http://code.google.com/soc/}} Since the lexicons
had been developed using different frequency lists, and slightly
different tagsets, they have been further trimmed and updated to
synchronise their coverage.

%wiktionaries and Wikipedia, as well as an SETimes corpus\footnote{\url{http://opus.lingfil.uu.se/SETIMES.php}} (\citealp{tyers2010south}) and a%corpus composed from the Serbian, Bosnian, Croatian and Serbo-Croatian Wikipedias.

%% Other resources for morphological analysis of Serbian and Croatian exist
%% (\citealp{vitas2004intex}, \citealp{vitas2003processing}, \citealp{agic2008improving}, \citealp{snajder08automatic}), 
%% to our knowledge there are none freely available for either Serbian, Bosnian or
%% Croatian. 
